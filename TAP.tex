\documentclass[11pt,a4paper]{ivoa}
\input tthdefs

\usepackage[utf8]{inputenc}

\title{Table Access Protocol}

\ivoagroup{Data Access Layer Working Group}

\author{Patrick Dowler}
\author{TBD}

\editor{Patrick Dowler}

\previousversion[http://www.ivoa.net/Documents/TAP/1.0]{TAP-1.0}
       

\begin{document}

\begin{abstract}
The table access protocol (TAP) defines a service protocol for accessing general 
table data, including astronomical catalogs as well as general database tables. 
Access is provided for both database and table metadata as well as for actual 
table data. This version of the protocol includes support for multiple query 
languages, including queries specified using the Astronomical Data Query 
Language (ADQL [1]) and the Parameterised Query Language (PQL, under 
development) within an integrated interface. It also includes support for both 
synchronous and asynchronous queries. Special support is provided for spatially 
indexed queries using the spatial extensions in ADQL. A multi-position query 
capability permits queries against an arbitrarily large list of astronomical 
targets, providing a simple spatial cross-matching capability. More 
sophisticated distributed cross-matching capabilities are possible by 
orchestrating a distributed query across multiple TAP services.  
\end{abstract}


\section*{Acknowledgments}

The authors would like to acknowledge all contributors to this and previous 
versions of this standard, especially: K. Andrews, J. Good, R. Hanisch, G. 
Lemson, T. McGlynn, K. Noddle, F. Ochsenbein, I. Ortiz, P. Osuna, R. Plante, G. 
Rixon, J. Salgado, A. Stebe, and A. Szalay.


\section*{Conformance-related definitions}

The words ``MUST'', ``SHALL'', ``SHOULD'', ``MAY'', ``RECOMMENDED'', and
``OPTIONAL'' (in upper or lower case) used in this document are to be
interpreted as described in IETF standard, \citet{std:RFC2119}.

The \emph{Virtual Observatory (VO)} is general term for a collection of 
federated resources that can be used to conduct astronomical research, 
education, and outreach. The \href{http://www.ivoa.net}{International
Virtual Observatory Alliance (IVOA)} is a global collaboration of separately 
funded projects to develop standards and infrastructure that enable VO 
applications.


\section{Introduction}
The Table Access Protocol (TAP) is a Web-service protocol that gives access to 
collections of tabular data referred to collectively as a tableset.  TAP 
services accept queries posed against the tableset available via the service and 
return the query response as another table, in accord with the relational model. 
 Queries may be submitted using various query languages and may execute 
synchronously or asynchronously. Support for the Astronomical Data Query 
Language (ADQL, [1]) is mandatory; support for other query languages such as 
Parameterised Query Language (PQL, under development) or native SQL is optional.

The result of a TAP query is another table, normally returned as a VOTable. 
Support for VOTable output is mandatory; all other formats are optional.

The table collections made accessible via TAP are typically stored in relational 
database management systems (RDBMS). A TAP service exposes the database schema 
to client applications so that queries can be posed directly against arbitrary 
data tables available via the service.

Multi-table operations such as joins or cross matches are possible provided the 
tables are all managed by the local TAP service, and provided the service 
supports these capabilities.  Larger scale operations such as a distributed 
cross match are also possible, but require combining the results of multiple TAP 
services.

\subsection{Query Types}
TAP services support three kinds of queries: data queries, metadata queries, and 
Virtual Observatory Support Interface (VOSI [6]) queries.

\subsubsection{Data Queries}
Data queries apply to the astronomical content served by a TAP service. This is 
the reason for providing a TAP service. All the other kinds of query support the 
ability to make data queries. Data queries may be specified in any query 
language supported by the service.

\subsubsection{Metadata Queries}
Metadata queries work like data queries, using the same query languages, but 
they are applied to standardized tables that are a subset of, and patterned 
after, information schema in RDBMS; the content of these tables explains the 
data model of a particular TAP service. Metadata queries allow a client to 
discover the names of tables and columns to be used in data queries.

\subsubsection{VOSI}
Metadata queries work like data queries, using the same query languages, but 
they are applied to standardized tables that are a subset of, and patterned 
after, information schema in RDBMS; the content of these tables explains the 
data model of a particular TAP service. Metadata queries allow a client to 
discover the names of tables and columns to be used in data queries.

\subsection{Query Languages}
TAP supports the use of multiple query languages, some of which are described 
here.

\subsubsection{ADQL Queries}
Support for ADQL queries is mandatory. ADQL can be used to specify queries  that 
access one or more tables provided by the TAP service, including the standard 
metadata tables. In general, the client must access table metadata in order to 
discover the names of tables and columns and then formulate queries. ADQL 
queries provide a direct (low-level) access to the tables; a query will be 
written for a specific TAP service and will not be usable with other services 
unless the query refers only to common tables and columns. It is also possible 
that the service registration (in an IVOA Registry)  may include sufficient 
table metadata to enable queries to be written directly.

Details of the ADQL language may be found in [1].

\subsubsection{PQL}
Support for PQL is optional. PQL can be used to formulate queries that access a 
single table provided by the TAP service, including the standard set of metadata 
tables. PQL can also be used in some cases without first querying the metadata 
tables by using the PQL parameters which carry sufficient meaning to enable the 
service to decide which tables and columns to use (e.g. POS, SIZE, REGION, BAND, 
TIME).

Details of the PQL language (parameters) are not part of the TAP specification.

\subsubsection{Other Query Languages}
A TAP service may also support use of other query languages, including 
pass-through of native SQL directly to an underlying DBMS, by describing such 
capabilities in the service metadata and allowing custom values of the service 
parameters. This mechanism allows future developments within the VOQL Working 
Group and outside the IVOA to be used without revising the TAP specification.

\subsection{Query Execution}
The TAP service specification defines both synchronous and asynchronous query 
execution. Users select synchronous or asynchronous execution by chosing the 
appropriate resource below the base URL for the service (see 2.2 ). A query is 
synchronous if the results of the query are delivered in the HTTP response to 
the request that originally posed the query. If the service returns an immediate 
HTTP-response upon accepting a query and the client later obtains the results of 
the query in response to a separate HTTP request, then we say the request is 
asynchronous.

\subsubsection{Asynchronous Queries}
Asynchronous query support is mandatory. Asynchronous queries require that 
client and server share knowledge of the state of the query during its execution 
and between HTTP exchanges.  They are an example of stateful interactions.  In 
TAP, the mechanism by which the clients and services share the state of 
transactions is based on the Universal Worker Service (UWS) pattern [3].

\subsubsection{Synchronous Queries}
Synchronous query support is mandatory. Synchronous queries execute immediately 
and the client must wait for the query to finish.  If the HTTP request times out 
or the client otherwise loses the connection to the service before receiving the 
response, then the query fails.

Synchronous query execution is adequate when the query will execute quickly and 
with a small number of results, or when they can at least start returning 
results quickly. They are generally simple to implement using standard web 
technologies and easy to use from a browser or scripting environment. However, 
synchronous requests are generally not sufficient and are likely to fail for 
queries that take a long time to execute, especially before returning any 
results.

\subsection{Interface Overview}
Table Access Protocol (TAP) is implemented over the HTTP protocol using standard 
HTTP GET and POST requests and conventions. A TAP request specifies one or more 
parameter key/value pairs; both keys and values are strings. The keys used are 
discussed in this specification and in the specifications for query languages 
supported by a service. The values may need to be encoded, using standard 
URL-encoding. For the following examples, http://example.com/tap/ is the base 
URL for a TAP service.

This is an example of a synchronous ADQL query on r magnitude:

\begin{verbatim}
HTTP POST htp://example.com/tap/sync
REQUEST=doQuery
LANG=ADQL
QUERY=SELECT * FROM magnitudes as m where m.r>=10 and m.r<=16
\end{verbatim}

NOTE: equivalent PQL that was in TAP-1.0 doc removed

Synchronous queries return the table of results in the HTTP response to the 
initial request. In the examples above, the output format defaults to VOTable; 
the FORMAT parameter could be added to select a different format.

Asynchronous queries are created in the same way as the synchronous kind, using 
the /async endpoint:

\begin{verbatim}
HTTP POST http://example.com/tap/async
REQUEST=doQuery
LANG=ADQL
QUERY=SELECT * FROM magnitudes AS m WHERE m.r>=10 AND m.r<=16
\end{verbatim}

NOTE: equivalent PQL that was in TAP-1.0 doc removed
 
The service's response to these requests is a URL representing the query's state 
and progress and where the state may be monitored and controlled. The query 
result or an error document can then be retrieved from a URL associated with the 
job. This is an application of the UWS pattern. The query is then executed 
with a separate request to run the job URL:

\begin{verbatim}
HTTP POST http://example.com/tap/async/<jobid>/phase
PHASE=RUN
\end{verbatim}

In addition to the sync and async resources for query execution, a TAP service 
also has metadata resources defined by the VOSI standard. The availability of a 
service can be monitored by accessing:

\begin{verbatim}
HTTP GET http://example.com/tap/availability
\end{verbatim}

See 2.2.3 for details of the availability resource.

The complete table metadata can be obtained:

\begin{verbatim}
HTTP GET http://example.com/tap/tables
\end{verbatim}

See 2.2.5 for details of the table metadata resource.

The capabilities can be obtained by:

\begin{verbatim}
HTTP GET http://example.com/tap/capabilities
\end{verbatim}

The capabilities are also accessible via a service request to the synchronous 
query resource:

\begin{verbatim}
HTTP GET http://example.com/tap/sync?REQUEST=getCapabilities
\end{verbatim}

This output lists support for optional TAP functionality and additional 
implemented interfaces. See 2.2.4 for details.


\subsection{Role within the VO Architecture}

NOTE: not in TAP-1.0

\begin{figure}
\centering

% Get the architecture diagram from the TCG chair
% http://wiki.ivoa.net/twiki/bin/view/IVOA/IvoaTCG
% If they give you a PDF, for now dumb it down to a png by
% convert -antialias -density 72x72 archdiag.pdf archdiag.png
% Oh -- Notes don't need this; you'd have to remove archdiag.png
% from FIGURES in the Makefile, too.

\includegraphics[width=0.9\textwidth]{archdiag.png}
\caption{Architecture diagram for this document}
\label{fig:archdiag}
\end{figure}

Fig.~\ref{fig:archdiag} shows the role this document plays within the
IVOA architecture \citep{note:VOARCH}.



\section{Requirements for a TAP Service}

NOTE: pre-amble about must, should, and may that was here in 1.0 removed since 
it is alteady included above

The TAP standard is identified using standardID="ivo://ivoa.net/std/TAP". This 
document specified version 1.0 of the standard.

\subsection{Feature Overview}
An implementation of a TAP service provides features as follows.

\begin{tabular}{l l l l l}
synchronous query execution & /sync & must & & \\
asynchronous query execution & /async & must & & \\
VOSI-availability & /availability & should & & \\
VOSI-capabilities & /capabilities & must & & \\
VOSI-tables & /tables & should & REQUEST=getCapabilities & must \\
ADQL queries & & & REQUEST=doQuery LANG=ADQL & must \\
PQL queries & & & REQUEST=doQuery LANG=PQL & may \\
other query languages & & & REQUEST=doQuery LANG=<other> & may \\
VOTable output & & & FORMAT & must \\
other formats & & & FORMAT & should \\
limiting output & & & MAXREC & must \\
logging & & & RUNID & should \\
\end{tabular}

The resources and parameters are described in detail below. The description of 
these resources and parameters spell out how the requirements here are to be 
implemented.

TAP service registration in the IVOA resource-registry is specified in section 
3.

\subsection{Resources}
A TAP service must be represented as a tree structure of web resources each 
addressable via a URL in the http scheme, or the https scheme, or both.

The web resource at the root of the tree must represent the service as a whole. 
This specification defines no standard representation for this root resource. 
Implementations may provide a representation, or may return a '404 not found' 
response to requests for the root web-resource. One possible representation is 
an HTML page describing the scientific usage and content of the service. TAP 
clients must not depend on a specific representation of the root web-resource.

\subsubsection{/sync}
A TAP service must provide a web resource with relative URL /sync that is a 
direct child of the root web resource. This web resource represents the results 
of synchronous requests. The exact form of the query, and hence the 
representation of the resource, is defined by the  query parameters as listed in 
section 2.3. Representations of results of queries and VOSI outputs are defined 
in sections 2.7.1 and 2.7.2 respectively.

For query languages that produce a single result (e.g. ADQL) executed using the 
/sync endpoint, the result of a successful query is returned in the response or 
the response includes an HTTP redirect (303: See Other) to a resource (uri) from 
which the result may be retrieved.

An HTTP-GET request to the /sync web resource may return a cached copy of the 
representation. This cached copy might come from an HTTP cache between the 
client and the service, and the service may also maintain its own cache. Clients 
which require an up-to-date representation of volatile data or metadata must use 
HTTP POST.

\subsubsection{/async}
A TAP service must provide a web resource with relative URL /async that is a 
direct child of the root web resource. This web resource represents controls for 
asynchronous queries. Specifically, the web resource must represent the job-list 
as specified in the UWS standard [3].

The child web resources of the /async resource are as specified by UWS. These 
are descendants of the /async web-resource, and they include a web resource that 
represents the eventual result of an asynchronous query, e.g.:
\begin{verbatim}
http://example.com/tap/async/42/results/result
\end{verbatim}
where the base URL for the TAP service is:
\begin{verbatim}
http://example.com/tap
\end{verbatim}
the UWS job list is:
\begin{verbatim}
http://example.com/tap/async
\end{verbatim}
and the job resource is
\begin{verbatim}
http://example.com/tap/async/42
\end{verbatim}
where 42 is the job identifier. A client making an asynchronous request must use 
the UWS facilities to monitor or control the job. In addition to the job list 
and job resource above, UWS specifies the name and semantics of the a small set 
of child resources used to view and control the job, e.g.:
\begin{verbatim}
http://example.com/tap/async/42/phase
http://example.com/tap/async/42/quote
http://example.com/tap/async/42/executionduration
http://example.com/tap/async/42/destruction
http://example.com/tap/async/42/error
http://example.com/tap/async/42/parameters
http://example.com/tap/async/42/results
http://example.com/tap/async/42/owner
\end{verbatim}
Successful TAP queries produce results which must be accessible as  resources 
under the UWS result list, e.g.:
\begin{verbatim}
http://example.com/tap/async/42/results/
\end{verbatim}
Failed TAP queries produce an error document (see 2.9 ) which must be accessible 
as the error resource, e.g.:
\begin{verbatim}
http://example.com/tap/async/42/error
\end{verbatim}
For query languages that produce a single result executed using the /async 
endpoint, the result of a successful query can be found within the result list 
specified by UWS [3]; the result must be named result and thus clients are able 
to access it directly, e.g.:
\begin{verbatim}
http://example.com/tap/async/42/results/result
\end{verbatim}
Access of this resource must deliver the result, either directly or as an HTTP 
redirect (303: See Other) to a resource from which the result may be retrieved.

For query languages that may produce multiple result resources, the names of the 
results are not specified (they may be specified in the specification for the 
language). The client can always access the result list resource as specified by 
UWS [3].

If the query returned no rows, the result resource must exist and contain no 
data rows. Details on interacting with these resources are specified in the UWS 
standard; for examples specific to TAP see Section 5 below.

\subsubsection{/availability}
The VOSI availability metadata should be accessible from a web resource with 
relative URL /availability that is a direct child of the root web resource. If 
implemented, the /availability resource must be accessible via the http GET 
method. The content is described by [6].

Services which do not implement the /availability resource must respond with an 
HTTP response code of 404 when this resource is accessed.

\subsubsection{/capabilities}
The service capabilities must be accessible from a web resource with relative 
URL /capabilities that is a direct child of the root web resource. The 
/capabilities resource must be accessible via the http GET method. The content 
is described by [8].

\subsubsection{/tables}
The table metadata should be accessible from a web resource with relative URL 
/tables that is a direct child of the root web resource. The /tables resource 
must be accessible via the http GET method.  The content is described by [7] and 
is equivalent to the metadata from the TAP\underline{' '}SCHEMA described in 2.6 
.

Services which do not implement the /tables resource must respond with an HTTP 
response code of 404 when this resource is accessed.

\subsection{Parameters}
The /sync and /async web-resources must accept the parameters listed in the 
following sub-sections. In a synchronous request, the parameters select the 
representation returned in the response message. In an asynchronous request, the 
parameters select the representation of the eventual query result rather than 
the response to the initial request.

Requirements on the presence and values of parameters described below are 
enforced only when the TAP request is executed (not when individual HTTP 
requests are handled). Thus, for asynchronous TAP queries, the parameter 
requirements must be satisfied (and errors returned if not) only when the query 
is run in (in the sense of UWS job execution). Specifically, asynchronous 
queries may be created with with no parameters and multiple, subsequent HTTP 
POST actions may specify the parameters in any order.

Not all combinations of the parameters are meaningful. For example, if a request 
carries  LANG=ADQL then the SELECT parameter (from PQL) is spurious. If a 
service receives a spurious parameter in an otherwise correct request, then the 
service must ignore the spurious parameter, must respond to the request normally 
and must not report errors concerning the  spurious parameter.

\subsubsection{REQUEST}
This parameter distinguishes current service operations, makes it possible to 
extend the service specification (with additional or custom operations), and 
specifies how other parameters should be interpreted. If a TAP service attempts 
to execute a TAP request without this parameter or with an incorrect value for 
this parameter, then the service must reject the request and return an error 
document as the result.

These are the standard values of the parameter:

doQuery: execute a query 

getCapabilities: return VOSI-capabilities metadata 

All requests to execute (/async or /sync) a query using a query language must 
include REQUEST=doQuery and must include the LANG parameter. For other values of 
REQUEST, additional parameters may or may not be required. The 
REQUEST=getCapabilities service operation must be supported for synchronous 
(/sync) requests and is not defined for asynchronous (/async) requests.

For synchronous queries, the HTTP request must also include additional 
parameters (see below) with the details of the query. These are used for 
metadata queries and data queries.

For asynchronous queries, the additional parameters may be included with the 
HTTP request that creates the query (the UWS job) or they may be POSTed directly 
to the created job resource, in one or more separate HTTP requests. The 
parameter names remain the same in both cases.

\subsubsection{VERSION}
The VERSION parameter specifies the TAP protocol version number. The format of 
the version number, and version negotiation, are described in section 2.9.2 .

A TAP service must support the VERSION parameter. This specification is for TAP 
1.0; the client would specify VERSION=1.0 if the request is made using the 
protocol described here.

\subsubsection{LANG}
The LANG parameter specifies the query language. The service must support LANG 
and the client must provide a value with REQUEST=doQuery. The only standard 
values for the LANG parameter are ADQL (a required language) and PQL (reserved 
value for an optional language which is under development). Support for other 
languages and the LANG value to use with them is described in the service 
capabilities.

For example, an ADQL query would be performed with
\begin{verbatim}
REQUEST=doQuery
LANG=ADQL
QUERY=<ADQL query string>
\end{verbatim}
A PQL query would be performed with
\begin{verbatim}
REQUEST=doQuery
LANG=PQL
<PQL-specific parameters>
\end{verbatim}
The value of LANG is a string specifying the language and optionally the 
language version used for the query parameter(s), as defined by the service 
capabilities.  The client may specify the version of the query language,  e.g. 
LANG=ADQL-2.0 (the syntax should be as shown) or it may omit the version, e.g. 
LANG=ADQL.  The service should return an “unknown query language” error as 
described in 2.9 if an unsupported language or an incompatible language version 
is specified.

\subsubsection{QUERY}
The QUERY parameter is used to specify the ADQL query. It may also be used to 
specify the query for other values of LANG (e.g. LANG=<some RDBMS-specific SQL 
variant>) which are not specified in this document but may be described in the 
service capabilities.

A service must support the QUERY parameter because ADQL is a required language.  
The case sensitivity of the query string is defined solely by the query language 
specification. In the case of ADQL 2.0, for example, the query is not case 
sensitive except for character  literals; schema, table, and column names, 
function names, and other ADQL keywords are not case sensitive.

Within the ADQL query, the service must support the use of timestamp values in 
ISO8601 format, specifically yyyy-MM-dd['T'HH:mm:ss[.SSS]], where square 
brackets denote optional parts and the 'T' denotes a single character separator 
(T) between the date and time parts.

If the tables that are queried through a service contain columns with spatial 
coordinates and the service supports spatial querying via the ADQL “region” 
constructs, the service must support the INTERSECTS function and it must support 
the following geometry functions: REGION, POINT, BOX, CIRCLE, COORD1, COORD2, 
COORDSYS. Support for the AREA, CONTAINS, and POLYGON functions are optional. If 
the service supports the REGION function, it must support region encoding in 
STC-S format (see section 6 ); the extent of STC-S support within the REGION 
function is left up to the implementation. Coordinate system specification for 
POINT, BOX, CIRCLE, and POLYGON must use values from STC-S as described in 
section 6 .

Note: Although it is allowed by the ADQL syntax, clients should be careful when 
mixing constants and column references for coordinate system and coordinate 
values. For example, POINT('ICRS', t.ra, t.dec) does not cause t.ra and t.dec to 
be transformed to ICRS; it simply tells the service to treat the values  as 
being expressed in that coordinate system.

\subsubsection{Parameters for PQL}
A number of parameters will be defined by the PQL standard for use in parametric 
queries. All of the parameters for PQL are are used unchanged in TAP.

Within the PQL query, the service must support the use of timestamp values in 
ISO8601 format (see 2.3.4 ).

If the table that is queried contains columns with spatial coordinates and the 
service provider wants to enable the caller to perform spatial queries, the 
service must support the PQL spatial constraint parameters (POS,SIZE and 
REGION). If a service supports the REGION parameter, it  must support region 
encoding in STC-S as decribed in section 6 ; the extent of STC-S support within 
the REGION function is left up to the implementation. Coordinate system 
qualifiers must use values from from STC-S as described in section 6 .

PQL defines symbolic values (@something). In TAP these can be used to refer to  
an uploaded table (see 2.5 ) with parameters that support a table reference. 
When used in this way, the uploaded table must be treated as if in the 
TAP\underline{' '}UPLOAD schema (e.g. @TAP\underline{' '}UPLOAD.mytable). 
Details on how to use table references in PQL will be described in the PQL 
specification.

\subsubsection{FORMAT}
The FORMAT parameter indicates the client's desired format for the table of 
results of a query. Its value should be a MIME type for tabular data or one of 
the following shorthand forms:

TODO: add table here

Both MIME types and the shorthand forms are insensitive to case. If the FORMAT 
parameter is omitted, the default format is VOTable.

A TAP service must support VOTable as an output format, should support CSV and 
TSV output and may support other formats. A TAP service must accept a FORMAT 
parameter indicating a format that the service supports and should reject 
queries where the FORMAT parameter specifies a format not supported by the 
service implementation.

\subsubsection{MAXREC}
The service must accept a MAXREC parameter specifying the maximum number of 
table records (rows) to be returned. If MAXREC is not specified in a query, the 
service may apply a default value or may set no limit. If the result set for a 
query exceeds this value, the service must only return the requested number of 
rows. If the result set is truncated in this fashion, it must include an 
overflow indicator as specified in section 2.7.4 .

The service must support the special value of MAXREC=0. This value indicates 
that, in the event of an otherwise valid request, a valid output table be 
returned containing metadata, no table data rows, and an overflow indicator as 
specified in section 2.7.4 .  The service is not required to execute the query; 
a successful  MAXREC=0 request does not necessarily mean that the query is valid 
and the overflow indicator does not necessarily mean that there is at least one 
row satisfying the query. The service may perform validation and may try to 
execute the query, in which case a MAXREC=0 request can fail.

A query with MAXREC=0 can be used with a simple query (e.g. SELECT * FROM  
some\underline{' '}table) to extract and examine the VOTable metadata (assuming 
FORMAT=votable). Note: in this version of TAP, this is the only mechanism to 
learn some of the detailed metadata, such as coordinate systems used.

\subsubsection{RUNID}
The service should implement the RUNID parameter, used to tag service requests 
with the job ID of a larger job of which the request may be part. The RUNID 
parameter is defined in [3] for /async requests; services should also implement 
it for /sync requests.

For example, if a cross match portal issues multiple requests to remote TAP 
services to carry out a cross-match operation, all would receive the same RUNID, 
and the service logs could later be analyzed to reconstruct the service 
operations initiated in response to the job.

The service should ensure that RUNID is preserved in any service logs and  
should pass on the RUNID value in any calls to other services.

\subsubsection{UPLOAD}
The service should support table upload via the UPLOAD parameter. The value is a 
list of table-name,URI pairs. Table names must be legal ADQL table names as 
defined in [1]. URIs maybe be simple URLs (e.g. with a URI scheme of http) or 
URIs (e.g. with a URI scheme of vos or param) that must be resolved to give the 
location of the table content. See section 2.5 for details.

\subsubsection{Case of Parameters}
NOTE: this is in DALI now

\subsubsection{Order and Cardinality of Parameters}
NOTE: this is in DALI now

\subsection{Table Names}
A fully qualified table name has the form
\begin{verbatim}
[[catalog_name”.”]schema_name”.”]table_name
\end{verbatim}
where catalog\underline{' '}name is the name of the DB catalogue (often the 
“database” name) in SQL DBMS terminology, schema\underline{' '}name is the name 
of the “schema” in DBMS terminology (often also called a “database”; a DBMS 
schema is a type of data model where the top level data model elements are 
tables), and table\underline{' '}name is the actual table name.  All elements of 
the table name are optional except table\underline{' '}name.  Depending upon the 
DBMS, “catalog” or “schema” may or may not be implemented; some DBMS implement 
both, others one or the other, and the simplest database systems might not 
implement either.

The implementation of a TAP service must define the table names acceptable in 
queries and must reveal these to clients through metadata queries or through 
VOSI-tables output, and the names must be identical in each of these sources. A 
TAP client must determine the acceptable names from one of these sources or from 
the cached form of the VOSI-tables output included in the service's 
registration. 

\subsection{Table Upload}
The service should implement the table upload capability. If upload is 
supported,  the service must accept tables in VOTable format. The client 
specifies the name of the uploaded table; this name must be a legal ADQL table 
name with no catalog or schema (e.g. an unqualified table name). Uploaded tables 
must be referred to in queries as TAP\underline{' '}UPLOAD.<tablename>, where 
<tablename> is the specified by the user.

Tables in the TAP\underline{' '}UPLOAD schema are transient and 
persist only for the lifetime of the query (although caching might be used 
behind the scenes) and are never visible in the TAP\underline{' '}SCHEMA 
metadata.

The column names in the transient database table are taken directly from the 
name attribute of the VOTable FIELD and PARAM elements. The datatypes of the 
transient table are determined from the FIELD and PARAM attributes as follows:

TODO: table

The default mapping of data types are shown above (no arraysize or xtype). If 
the xtype attribute is set, this is the preferred internal datatype. If xtype is 
not set, then the datatype and arraysize indicate the most suitable internal 
datatype.

In the arraysize column above, [1] means the arraysize is not set or is set to 
1, n means arraysize is set to a specific value, * means arraysize=”*”, and n* 
means arraysize=”n*” (variable size up to length n). A blank means the arraysize 
is not set.

Binary values (unsignedByte in VOTable, BINARY, VARBINARY, or BLOB in ADQL) can 
be expressed as specified by the VOTable standard. By default, VOTable allows 
them to be written as an array of decimal numbers, e.g. 12 56 0 255 0 0 255 (one 
number per byte value).

For columns of type BLOB or CLOB, most database systems support reference to 
these columns in the select clause but not in any other part of the query. 
Services may use these types to indicate that columns may only be selected. For 
example, if service implementors want to make URL(s) available as column values 
in the results, but do not actually store the URL(s) in the database, they would 
specify a column with xtype=”adql:CLOB” and the column with URL(s) could be 
referenced in the SELECT clause of a query, but could not be used in the WHERE 
clause. The service could then process the query result and insert the URL(s) 
or, more likely, transform a column value (an identifier) into a URL while 
writing the results.

TIMESTAMP values are specified using ISO8601 format without a timezone (as in 
2.3.4 ) and are assumed to be in UTC. The xtype=”adql:TIMESTAMP” attribute must 
be specified in an uploaded VOTable in order for the values to be inserted in a 
column of type TIMESTAMP; without the xtype, the values would be inserted into a 
CHAR(n) or VARCHAR column.

POINT and REGION values are specified in STC-S format (see section 6 ). The 
xtype=”adql:POINT” attribute must be specified in an uploaded VOTable in order 
for the char values to be parsed and treated as POINTs (e.g. to be used with 
some of the ADQL region functions). For regions, the xtype=”adql:REGION” 
attribute must be specified in an uploaded VOTable in order for the char values 
to be parsed and treated as REGIONs (e.g. to be used with some of the ADQL 
region functions).

\subsubsection{UPLOAD}
The UPLOAD parameter is used to reference read-only external tables via their 
URI, to be uploaded for use as input tables to the query.   The value of the 
UPLOAD parameter is a list of table name-URI pairs. Elements of the list are  
delimited by semicolon and the two parts of the pair are delimited by comma. For 
example:
\begin{verbatim}
UPLOAD=table_a,http://host_a/path;table_b,http://host_b/path
\end{verbatim}

would define two input tables table\underline{' '}a and 
table\underline{' '}b, located at the given URIs. Services that implement 
UPLOAD must support http as a URI scheme (e.g. must support treating an http URI 
as a URL). A VOSpace URI (vos:<something>)  is a more generic example of a URI 
that requires more service-side functionality; support for the vos scheme is 
optional.

\subsubsection{Inline Table Upload}
To upload a table inline, the caller must specify the UPLOAD parameter (as 
above) using a special URI scheme “param”. This scheme indicates that the value 
after the colon will be the name of the inline content. The content type used is 
multipart/form-data, using a “file” type input element. The “name” attribute 
must match that used in the UPLOAD parameter.

For example, in the POST data we might have this parameter:
\begin{verbatim}
UPLOAD=table_c,param:table1
\end{verbatim}
and this content:
\begin{verbatim}
Content-Type: multipart/form-data; boundary=AaB03

[...]

--AaB03x

Content-disposition: form-data; name="table1"; filename="table1.xml"

Content-type: application/x-votable+xml

[...]

--AaB03x

[...]
\end{verbatim}
The uploaded table would be referenced in queries as 
TAP\underline{' '}UPLOAD.table\underline{' '}c (the table name in the UPLOAD 
parameter). Services that implement table upload must support the param scheme 
for inline uploads.

In principle, any number of tables can be uploaded using the UPLOAD parameter 
and any combination of URI schemes supported by the service as long as they are 
assigned unique table names within the query. Services may limit the size and 
number of uploaded tables; if the service refuses to accept the entire table it 
must respond with an error as described in 2.7.3 .

\subsection{Metadata and TAP SCHEMA}
There are several approaches to getting metadata for a given TAP service. All 
TAP services must support a set of tables in a schema named 
TAP\underline{' '}SCHEMA that describe the tables and columns included in the 
service. In addition to the TAP\underline{' '}SCHEMA, there are two other ways 
to get metadata from a TAP service. First, the VOSI tables resource provides 
metadata on all tables and columns; this resource is described in 2.2.5 . The 
VOSI tables resource provides the same metadata as the TAP\underline{' '}SCHEMA 
but in a rigorously controlled format; the information in the 
TAP\underline{' '}SCHEMA is equivalent to that defined by the  VODataService 
[7]. Second, the client may specify a query of one or more tables setting the 
MAXREC parameter to 0 so that only the metadata regarding the requested fields 
is returned. Use of MAXREC is described in 2.3.7 .

The TAP\underline{' '}SCHEMA provides access to table, column, and join key 
metadata through the TAP query mechanisms themselves. Users can discover tables 
or columns that meet their specific criteria by querying the tables described 
below.  The service may enhance the TAP\underline{' '}SCHEMA with additional 
metadata where that seems appropriate; since it is self-describing, the 
TAP\underline{' '}SCHEMA may be queried to determine if any extended schema 
metadata is defined by the service. Services must provide these tables and make 
them accessible by all supported query mechanisms.

The qualified names in the tables of the TAP schema must follow the rules 
defined in section 2.4. The names must be stated in a form that is acceptable as 
an operand of a query.

All columns in the TAP\underline{' '}SCHEMA tables are of type VARCHAR except 
for size,  principal, indexed, and std (in Columns) which are INTEGER values.

Implementors are permitted to include additional tables in the 
TAP\underline{' '}SCHEMA to describe additional aspects of their service not 
covered by this specification. Implementors may also include additional columns 
in the standard tables described below. For example, one could include a column 
with a timestamp saying when metadata values were was last modified.

\subsubsection{Schemas}
The table TAP\underline{' '}SCHEMA.schemas must contain the following columns:

TODO: table

The schema\underline{' '}name values must be unique and may be qualified by the 
catalog name or not depending on the implementation requirements. The fully 
qualified schema name is defined by the ADQL language and  follows the pattern 
[catalog.]schema. The schema metadata are included for reference and are not 
used directly to construct queries.

\subsubsection{Tables}
The table TAP\underline{' '}SCHEMA.tables must contain the following columns:

TODO: table

The table\underline{' '}name values must be unique. The value of the 
table\underline{' '}name should be the string that is recommended for use in 
querying the table; it may or may not be qualified by schema and catalog name(s) 
depending on the implementation requirements. The fully qualified table name is 
defined by the ADQL language and follows the pattern [[catalog.]schema.]table. 

\subsubsection{Columns}
The table TAP\underline{' '}SCHEMA.columns must contain the following columns:

TODO: table

The table\underline{' '}name,column\underline{' '}name (pair) values must be 
unique.

Data types and how they map to VOTable datatypes are described in section 2.5 
above. The “size” gives the length of variable length datatypes, for example 
varchar(256); this size does not map to the VOTable arraysize attribute when the 
latter specifies the size and shape of a multi-dimensional array. The 
“principal” flag indicates that the column is considered a core part the 
content; clients can use this hint to make the principal column(s) visible, for 
example by selecting them by default in generating an ADQL query. In cases where 
the services selects the columns to return (such as PQL without a SELECT 
parameter), the principal column indicates those columns that are returned by 
default. The “indexed” flag indicates that the column is indexed, potentially 
making queries run much faster if this column is used in a constraint. The “std” 
is included for compatibility with the registry, which uses this value to 
indicate that a given column is defined by some standard, as opposed to a custom 
column defined by a particular service.

\subsubsection{Foreign Keys}
The table TAP\underline{' '}SCHEMA.keys must contain the following columns to 
describe foreign key relations between tables:

TODO: table

The key\underline{' '}id values are unique and used only to join with the 
TAP\underline{' '}SCHEMA.key\underline{' '}columns table below. There may be 
one or more rows with different key\underline{' '}id values and a pair 
of tables to denote one or more ways to join the tables.

The table TAP\underline{' '}SCHEMA.key\underline{' '}columns must contain the 
following columns to describe the columns that make up a foreign key:

TODO: table

There may be one or more rows with a specific key\underline{' '}id to 
denote single or multi-column keys.

A TAP service must provide the tables listed above and may provide other tables 
in the TAP\underline{' '}SCHEMA namespace.

\subsection{Access to and Representation of Results}
\subsubsection{Data and Metadata Queries}
The result of a data query or a metadata query depends on the query language 
used and may be one or more tables in one or more resources. Unsupportable 
combinations of query result and FORMAT (e.g. queries that produce multiple 
tables and an inherently single-table format like CSV) will cause the request to 
fail. Currently, an ADQL query result must be a single table (in a single file).

This table must be encoded in the output format specified by the FORMAT 
parameter of the query. See section 2.3.6 for required, optional and default 
formats. VOTable is the default format and VOTable support is mandatory.

The output table must include the same number and order of columns as specified 
in the SELECT clause of the query. For VOTable output, the name attribute of 
FIELD elements must be the same as the column names (or aliases if specified in 
the query) from the query and the datatype, arraysize, and xtype attributes of 
FIELD elements must be set using the mapping specified in section 2.5 . The 
xtype attribute in the output must match the datatype for the column in the 
TAP\underline{' '}SCHEMA.

VOTable structure follows the rules in section 2.9 and must be returned with an 
allowed VOTable MIME type (application/x-votable+xml or text/xml). If the FORMAT 
parameter (see 2.3.6 ) of the request specified a specific VOTable MIME type, 
the requested MIME type must be used in the HTTP response.

CSV formatted data should represent the output table with one row of text per 
table row, with the table column values rendered as text and separated by 
commas. If a column value contains a comma the entire column value should be 
enclosed in double quotes.  Text lines may be arbitrarily long.  The first data 
row should give the column name as the data value.   CSV data must be returned 
with a MIME type of text/csv; if the optional header line (with column names) is 
included, the MIME type must be text/csv;header=present. Full details of CSV 
format are defined in RFC 4180 [14].

TSV formatted data should represent the output table with one row of text per 
table row, with the table column values rendered as text and separated by the 
TAB character. TSV data must be returned with a MIME type of 
text/tab-separated-values [15]. Column values may not contain the TAB 
character. 

\subsubsection{VOSI}
Representations of VOSI outputs (capabilities, availability, table metadata) 
must be as defined in the VOSI standard [6].

The representation of table metadata must include all tables in the service's 
tableset. VOSI's representation of table metadata is specified in VODataService 
[7].

The VOSI standard specifies that the capability metadata is encoded as an XML 
document which lists each of the service's capabilities as a <capability> 
element. The type of this element (which defines the contents) is 
{http://www.ivoa.net/xml/VOResource/v1.0}Capability from the VOResource XML 
standard [8].

In addition, the capabilities output must also comply with the following    
requirements:

    •.the returned document must include one <capability> element that describes 
the service's support for the TAP protocol 

    •.this <capability> element must have its "standardID" attribute set to 
"ivo://ivoa.net/std/TAP" 

    •.this capability element must include at least one "<interface>" element 
with its "role" attribute set to "std",  

    •.this "<interface>" element must contain a child "<accessURL>" element with 
the attribute "use" set to "base" which contains the root web resource for the 
service as defined in section 2.2 . 

Note: VO registries recognize a service's support for a standard protocol 
through this capability description. In particular, a different standard 
Capability sub-type is used for each standard protocol to provide capability 
metadata that is specific to that protocol. At the time of this writing, a 
Capability sub-type for TAP has not yet been defined. Thus for compliance with 
this standard, any legal Capability description that meets the above 
restrictions is sufficient. However, once a VOResource extension for TAP is 
standardized, it is strongly recommended that TAP services emit its capabilities 
using that the Capability sub-type specialized for TAP.

For example, the returned capabilities document for a service supporting    TAP 
might look as follows:

\begin{verbatim}
<?xml version="1.0" encoding="UTF-8"?>

<vosi:capabilities xmlns=""

   xmlns:vosi="http://www.ivoa.net/xml/VOSI/v1.0"

   xmlns:vs="http://www.ivoa.net/xml/VODataService/v1.0"

   xmlns:xsi="http://www.w3.org/2001/XMLSchema-instance"

   xsi:schemaLocation="http://www.ivoa.net/xml/VOSI/v1.0

                       http://www.ivoa.net/xml/VOSI/v1.0

             http://www.ivoa.net/xml/VODataService/v1.0

             http://www.ivoa.net/xml/VODataService/v1.0">

  <vosi:capability standardID="ivo://ivoa.net/std/TAP">

    <interface xsi:type="vs:ParamHTTP" role="std">

      <accessURL use="base"> http://myarchive.net/myTAP </accessURL>

    </interface>

  </vosi:capability>

  <vosi:capability standardID="ivo://ivoa.net/std/VOSI#capabilities">

    <interface xsi:type="vs:ParamHTTP">

      <accessURL use="full">

        http://myarchive.net/myTAP/capabilities </accessURL>

    </interface>

  </vosi:capability>

  <vosi:capability standardID="ivo://ivoa.net/std/VOSI#availability">

    <interface xsi:type="vs:ParamHTTP">

      <accessURL use="full">

        http://myarchive.net/myTAP/availability

      </accessURL>

    </interface>

  </vosi:capability>

  <vosi:capability standardID="ivo://ivoa.net/std/VOSI#tables">

    <interface xsi:type="vs:ParamHTTP">

      <accessURL use="full">

        http://myarchive.net/myTAP/tables </accessURL>

    </interface>

  </vosi:capability>

</vosi:capabilities>
\end{verbatim}

\subsubsection{Errors}
If the service detects an exceptional condition, it must return an error 
document with an appropriate HTTP-status code. TAP distinguishes three classes 
of exceptions.

Errors in the use of the HTTP protocol. 

Errors in the use of the TAP protocol, including both invalid requests and 
failure of the service to complete valid requests. 

Error documents for HTTP-level errors are not specified in the TAP protocol. 
Responses to these errors are typically generated by service containers and 
cannot be controlled by TAP implementations. There are several cases where a TAP 
service could return an HTTP error. First, the /async endpoint could return a 
404 (not found) error if the client accesses a job within the UWS joblist that 
does not exist. Second, access to a resource could result in an HTTP 401 (not 
authorized) error if authentication is required or an HTTP 403 (forbidden) error 
if the client is not allowed to access the resource.

Error documents for TAP errors must be VOTable documents;  any result-format 
specified in the request is ignored. If the error document is being retrieved 
from the /async/<jobid>/error resource (specified by UWS) after an asynchronous 
query, the HTTP status code should be 200. If the error document is being 
returned directly after a synchronous query, the service may use an appropriate 
HTTP status code, including 200 (successfully returning a response to the 
request) and various 4xx and 5xx values. The exception condition must be 
described to the client using a status code in the VOTable header.  Section   
2.9 specifies the use of VOTable for error documents in TAP services. 

\subsubsection{Overflows}
If a query is executed by a TAP service, the number of rows in the table of 
results may exceed a limit requested by the user (using the MAXREC parameter) or 
a limit set by the service implementation (the default or maximum value of 
MAXREC). In these cases, the query is said to have 'overflowed'. Typically, a 
TAP service will not detect an overflow until some part of the table of results 
has been sent to the client.

If an overflow occurs, the TAP service must produce a table of results that is 
valid, in the required output format, and which contains all the results up to 
the point of overflow. Since an output overflow is not an error condition, the 
MIME type of the output must be the same as for any successful query and the 
HTTP status-code must be as for a successful, complete query.

If the output format is VOTable, section 2.9.1 describes the method by which the 
overflow is reported. No method of reporting an overflow is defined for formats 
other than VOTable.

\subsection{Versioning of the TAP Protocol}
The TAP protocol provides explicitly for versioning of the interface in order to 
support version negotiation between a client and a service where one or both 
parties support more than one version of the protocol. The TAP version refers 
only to the TAP protocol; query languages are versioned separately and TAP and 
ADQL versions may differ.

Version numbers follow IVOA document conventions [17].

\subsubsection{Appearance in requests and in service metadata}
The version number may appear in at least three places: in the service metadata, 
as a parameter in client requests to a server, and in the query response. The 
version number used in a client’s request of a particular server must be equal 
to a version number which that server has declared it supports (except during 
negotiation, as described below). A server may support several versions, whose 
values clients may discover according to the negotiation rules.

\subsubsection{Version number negotiation}
If a TAP client does not specify the version number in a request, the server 
assumes the highest standard version supported by the service, and no explicit 
version checking takes place.   If the client specifies an explicit version 
number, and this does not match a version available from the service, the 
service returns a version number mismatch error as described in 2.9.2 . The 
client can determine what versions of the protocol the service supports by a 
prior call to VOSI-capabilities or via a registry query.

\subsection{Use of VOTable}
VOTable is a general format. TAP requires that it be used in a particular way.

The result VOTable document must comply with VOTable v1.2 or greater [9]. For 
columns containing coordinate values, the coordinate system metadata should be 
provided as described in [13].

The VOTable must contain a RESOURCE element identified with the attribute 
type="results", containing a single TABLE element with the results of the query. 
Additional RESOURCE elements may be present, but the usage of any such elements 
is not defined here and TAP clients should not depend upon them.

\subsubsection{INFO elements}
The RESOURCE element must contain, before the TABLE element, an INFO element 
with attribute name = "QUERY\underline{' '}STATUS". The value attribute must 
contain one of the following values:

“OK”, meaning that the query executed successfully and a result table is 
included in the resource 

"ERROR”, meaning that an error was detected at the level of the TAP 
protocol or the query failed to execute 

The content of the INFO element conveying the status should be a message 
suitable for display to the user describing the status.

\begin{verbatim}
<INFO name="QUERY_STATUS" value="OK"/>
\end{verbatim}
 
\begin{verbatim}
<INFO name="QUERY_STATUS" value="OK">Successful query</INFO>
\end{verbatim}

\begin{verbatim}
<INFO name="QUERY_STATUS" value="ERROR">
   value out of range in POS=45,91
</INFO>
\end{verbatim}

Additional INFO elements may be provided, e.g., to echo the input parameters 
back to the client in the query response (a useful feature for debugging or to 
self-document the query response), but clients should not depend on these. 

\begin{verbatim}
<RESOURCE type=”results”>
<INFO name="QUERY_STATUS" value="ERROR">
    unrecognized operation
</INFO>
<INFO name="SPECIFICATION" value="TAP"/>
<INFO name=”VERSION” value=”1.0”/>
<INFO name="REQUEST" value="doQuery"/>
<INFO name="baseUrl" value="http://webtest.aoc.nrao.edu/ivoa-dal"/>
<INFO name="serviceVersion" value="1.0"/
...
</RESOURCE>
\end{verbatim}

If an overflow occurs (result exceeds MAXREC), the service must close the table 
and append another INFO element to the RESOURCE (after the TABLE) with 
name=”QUERY\underline{' '}STATUS” and the value=”OVERFLOW”.
\begin{verbatim}
<RESOURCE type=”results”>
<INFO name="QUERY_STATUS" value="OK"/>
...
<TABLE>...</TABLE>
<INFO name="QUERY_STATUS" value="OVERFLOW"/>
</RESOURCE>
\end{verbatim}

In the above example, the TABLE should have exactly MAXREC rows.

If an error occurs while writing the rows of the VOTable, the service must close 
the table and append another INFO element to the RESOURCE, after the TABLE, with 
name=”QUERY\underline{' '}STATUS” and the value=”ERROR”.
\begin{verbatim}
<RESOURCE type=”results”>
<INFO name="QUERY_STATUS" value="OK"/>
...
<TABLE>...</TABLE>
<INFO name="QUERY_STATUS" value="ERROR" />
</RESOURCE>
\end{verbatim}
The content of these trailing INFO elements is optional and intended for users; 
client software should not depend on it.

Thus, one INFO element with name=”QUERY\underline{' '}STATUS” and value=”OK” or 
value=”ERROR” must be included before the TABLE. If the TABLE does not contain 
the entire query result, one INFO element with value=”OVERFLOW” or value=”ERROR” 
 must be included after the table.
2.9.2 Version Mismatch Errors

Errors due to version mismatch from either the VERSION parameter (TAP version) 
or specific version used in the LANG parameter (query language version) are 
specified using an INFO element with name=”QUERY\underline{' '}STATUS” and 
value=”ERROR” as described above.  

\section{Service Registration}
Publication of a service to the VO requires that it be registered with an IVOA 
registry, including describing the identity and capabilities of the service.

The resource document for a TAP service instance must be structured according to 
VOResource [8] using the sub-type CatalogService as defined in VODataService 
[7].

The resource document must include a capability element denoting the TAP 
interface and functions. This element must contain the URL for the root web 
resource (as defined in section 2.2 ). Clients would add to this URL /sync or 
/async as appropriate.

The resource document must contain capability elements for the 
VOSI-capabilities, VOSI-availability and VOSI-tables outputs. These must be 
formatted as in the VOSI standard [6].

The resource document should include the table metadata, except where the  
database-schema of the archive changes frequently. Where table metadata are 
provided, they must be represented as XML elements drawn from VODataService 
[7]. 

\section{Extended Capabilities}
The TAP service allows for optional extended capabilities and operations. 
Extensions may be defined within an information community when needed for 
additional functionality or specialization.  A generic client must not be 
required or expected to make use of such extensions.  Extended capabilities or 
operations must be defined by the service metadata. Extended capabilities 
provide additional metadata about the service, and may or may not enable 
optional new parameters to be included in operation requests.  Extended 
operations may allow additional operations to be defined.

A server must produce a valid response to the operations defined in this 
document, even if parameters used by extended capabilities are missing or 
malformed (i.e. the server must supply a default value for any extended 
capabilities it defines), or if parameters are supplied that are not known to 
the server.

Service providers must choose extension names with care to avoid conflicting 
with standard metadata fields, parameters and operations.

\section{Use of UWS}
The UWS pattern is specified in [3] and its application to TAP in section 
2.2.2. This section explains the exchange of messages between a TAP client and 
service when using UWS to run an asynchronous query.

Consider a TAP service at http://example.com/tap. TAP mandates that the 
asynchronous requests be directed to http://example.com/tap/async. This URL 
points to the list of 'jobs'; i.e. the list of queries currently or recently 
executed.

\subsection{Creating a Query}
To create a new query, the client POSTs a request to the job list:

\begin{verbatim}
HTTP POST http://example.com/tap/async
REQUEST=doQuery
LANG=ADQL
QUERY=SELECT TOP 100 * FROM foo
\end{verbatim}

The service then creates a job and assigns that job a name and a URL based on 
the name. Suppose that the name is 42, then the URL will be 
http://example.com/tap/async/42 because the jobs are always children of the job 
list. While the job is in the PENDING phase, additional parameters may be 
specified by additional POSTs to the job resource, for example:

\begin{verbatim}
HTTP POST http://example.com/tap/async/42
UPLOAD=mytable,http://a.b.c/mytable.xml
\end{verbatim}

After each such POST, the service issues an HTTP redirection to the job's URL, 
where the modified state may be accessed:

\begin{verbatim}
HTTP status 303 'See other'
Location: http://example.com/tap/async/42
\end{verbatim}

All TAP-specific parameters are stored using the paramList mechanism of UWS and 
are included in the XML representation of the job:
\begin{verbatim}
HTTP GET http://example.com/tap/async/42
\end{verbatim}
or directly from the parameters resource:
\begin{verbatim}
HTTP GET http://example.com/tap /async/42/parameters
\end{verbatim}
Individual parameters cannot be accessed as separate web resources.

The UWS pattern requires the following resources to describe and control the 
job:
\begin{verbatim}
http://example.com/tap/async/42/phase
http://example.com/tap/async/42/quote
http://example.com/tap/async/42/executionduration
http://example.com/tap/async/42/destruction
http://example.com/tap/async/42/results
http://example.com/tap/async/42/error
\end{verbatim}
The quote resource specifies the predicted completion time for the job (query), 
assuming it is started immediately. In practice, it is very hard to estimate the 
time a query will take; for TAP services it is recommended that this be set to 
the current time plus the maximum amount of time the query will be allowed to 
run (see termination below). 

The termination resource specifies the amount of time (in seconds) the job 
(query) will be allowed to run before being aborted by the service. The 
termination time is set by the service and can be read from the job or directly 
from the termination resource:

\begin{verbatim}
HTTP GET http://example.com/tap/async/42/executionduration
\end{verbatim}
The service may allow the client to change the termination:
\begin{verbatim}
HTTP POST http://example.com/tap/async/42/executionduration
TERMINATION=600
\end{verbatim}
The destruction resource specifies when the service will destroy the job. The 
service is only required to keep a job for a finite period of time, after which 
it may destroy the job, including the result. After this time, the client will 
receive an HTTP 404 'not found' status if it tries to get any information about 
the job. The destruction time of the job is chosen by the service and the client 
can read it from the job or directly from the destruction resource:
\begin{verbatim}
HTTP GET http://example.com/tap/async/42/destruction
\end{verbatim}
The service may allow the client to change the destruction time:
\begin{verbatim}
HTTP POST http://example.com/tap/async/42/destruction
DESTRUCTION=2008-11-11T11:11:11Z
\end{verbatim}

\subsection{Running a Query}
The phase URL shows the progress of the job. When the job is created by the 
service it will normally be set to PENDING, but might be set to ERROR if the 
service has rejected the job. If the phase is ERROR, then the error URL should 
lead to a an error document explaining the problem. If the phase is PENDING, 
then the client needs to commit the job for execution.

The client runs the job by posting to the phase URL:
\begin{verbatim}
HTTP POST http://example.com/tap /async/42/phase
PHASE=RUN
\end{verbatim}

The service replies with a redirection to the job URL
\begin{verbatim}
HTTP status 303 'see other'
Location: http://example.com/tap /async/42
\end{verbatim}
The phase will now have changed to either QUEUED or EXECUTING, depending on the 
service implementation. The client tracks the execution by polling the phase 
URL:
\begin{verbatim}
HTTP GET http://example.com/tap/async/42/phase
\end{verbatim}
A job in the  QUEUED or EXECUTING phase may be aborted by posting to the phase 
URL:
\begin{verbatim}
HTTP POST http://example.com/tap/async/42/phase
PHASE=ABORT
\end{verbatim}

When the query is complete, the phase changes to COMPLETED. The client then 
retrieves the result from the results list:
\begin{verbatim}
HTTP GET http://example.com/tap/async/42/results/result
\end{verbatim}
The client knows that the table of results is at the URL /result relative to the 
results list because the TAP protocol requires this naming. A generic UWS client 
could find the name of the result and retrieve it by examining either the job 
description:
\begin{verbatim}
HTTP GET http://example.com/tap/async/42
\end{verbatim}
or by looking specifically at the result list:
\begin{verbatim}
HTTP GET http://example.com/tap/async/42/results
\end{verbatim}
If the service cannot run the query, then the final phase is ERROR and there is 
no table of results. In this case, the client should expect an HTTP 404 'not 
found' status if it tries to retrieve the result. The client should look instead 
at the error resource to find out what went wrong:
\begin{verbatim}
HTTP GET http://example.com/tap/async/42/error
\end{verbatim}
If the job was aborted (by the client or the service), the final phase will be 
ABORTED and there is no table or results. As with errors, the client should look 
at the error resource to find out what went wrong.

The basic sequence can be executed from a web browser or from a shell script 
using the curl utility:
\begin{verbatim}
curl -d 'REQUEST=doQuery&LANG=PQL&POS=12,34&SIZE=0.5&FROM=foo' \
       http://example.com/tap/async

  [read Location header from curl output]

curl -d 'PHASE=RUN' http://example.com/tap/async/42

curl http://example.com/tap/async/42/phase

  [repeat until phase is COMPLETED]

curl http://example.com/tap/42/results/result
\end{verbatim}

\section{Use of STC-S in TAP}

NOT: not included from TAP-1.0 since it was informative and does not belong in 
this docuemnt.

\section{VOSpace Integration}
This version of TAP provides limited VOSpace integration, although better 
support for VOSpace is planned for a later version following further 
implementation experience. In this version, one may specify an upload table 
using a URI to a table stored in a VOSpace, e.g.:
\begin{verbatim}
HTTP POST http://example.com/tap/async/42
UPLOAD=mytable,vos://space/path/votable.xml
\end{verbatim}
The service would resolve the URI, contact the VOSpace, retrieve the table, and 
make it visible to the query as TAP\underline{' '}UPLOAD.mytable.

A future version of TAP may specify additional use and more integration with 
VOSpace.

\section{Use of HTTP}
Note: This section is in or belongs in DALI.

\appendix

\section{Changes from Previous Versions}

\subsection{Changes from TAP-1.0}


\bibliography{ivoatex/ivoabib}


\end{document}
